There will be big big consequences of publishing this paper. Will they be good or bad?
Making changes that hopefully work.
After running all $240$ instances on each one of the models ($480$ in total), we start by summarizing the overall performance of each model. The BLP model is able to find the optimal solution on $25\%$ of the instances ($61$ of the $240$ instances) while, the ALP model found optimal solutions on $68\%$ of them ($164$ instances out of $240$).

Taking a closer look at the number of instances solved optimally by the models and segregating the data by the number of jobs, we are ab. le to note how the ability of each model to reach an optimal solution decreased as the number of jobs increased. On one side, in the particular case of the BLP model, the very smallest job group ($10$ jobs) are solved optimally, while the groups of $30$ and $40$ jobs show no instances solved optimally within the 1-hour time limit. On the other side, the ALP model solves optimally $100\%$ of the instances in the smallest group while all the job groups have some percentage of instances solved optimally. Figure \ref{fig:f1} illustrates these observations in more detail.
\begin{figure}[h!]
\caption{Count of optimal and timed-out instances}
\label{fig:f1}
\centering
\includegraphics[angle=0,width=5in]{fig1.png}
\end{figure}

By using the computational time and the relative gap as measures of model performance, we compare the BLP and ALP models side by side. Measures of computational time include the number of instances that timed out (TO), as well as the minimum, average and maximum values for each job group from $10$ through $40$. Similarly the relative gap is summarized via the minimum, average and maximum values. This comparison is separated by aggregation group (groups 1 through group 4) and summarized in Tables \ref{tab:t5} through \ref{tab:t8}.\\
%%%%%%%%%%%%%%%%%%%%%%%%%%
\begin{table}[h!] \small
\renewcommand{\arraystretch}{1.4} %defines space between rows
\caption{Group 1 - No aggregation (60 instances)}
\label{tab:t5}
\centering {
\begin{tabularx}{\textwidth}{X|X|X|X|X|X|X|X||X|X|X|X|X|X|X|}
\cline{2-15}
& \multicolumn{7}{c||}{BLP} & \multicolumn{7}{c|}{ALP} \\
\cline{2-15}
& \multicolumn{4}{c}{Computational Time\cellcolor[gray]{0.9}} & \multicolumn{3}{|c||}{Relative Gap (\%)} & \multicolumn{4}{c|}{Computational Time\cellcolor[gray]{0.9}} & \multicolumn{3}{c|}{Relative Gap (\%)} \\
\hline
\multicolumn{1}{ |c|  }{Jobs} 	& TO & Min 	& Avg 	& Max 	& Min 	& Avg 	& Max 	& TO & Min 	& Avg 	& Max 	& Min 	& Avg 	& Max \\
\hline
\multicolumn{1}{ |c|  }{10}  	&0	 &0.047	& 1.657	& 9.268	&0.0 	&0.0	&0.0	&0 	& 0.016	& 0.323	& 1.137	&0.0	&0.0	&0.0  \\
\hline
\multicolumn{1}{ |c|  }{20}  	&11	 &10.29 & 3129	& 4788	& 0.0 	&25.2 	& 67.1	&4 	& 2.668	& 1043	& 3655	& 0.0	& 0.4	& 1.6 \\
\hline
\multicolumn{1}{ |c|  }{30}  	&15  &3622 	& 3867	& 4789 	& 31.6	&59.9 	& 74.3	&14 & 3.619	& 3494	& 4062	& 0.0	& 1.4	& 2.5  \\
\hline
\multicolumn{1}{ |c|  }{40}  	&15  &3600 	& 3744 	& 3901	& 41.0 	&63.7 	& 80.1	&15	& 3617 	& 3934	& 4226	& 0.8	& 2.0	& 4.6  \\
\hline
\end{tabularx}}
\end{table}\\
%%%%%%%%%%%%%%%%%%%%%%%%%%
\begin{table}[h!] \small
\renewcommand{\arraystretch}{1.4} %defines space between rows
\caption{Group 2 - 2 Jobs per customer on ALP model (60 instances)}
\label{tab:t6}
\centering {
\begin{tabularx}{\textwidth}{X|X|X|X|X|X|X|X||X|X|X|X|X|X|X|}
\cline{2-15}
& \multicolumn{7}{c||}{BLP} & \multicolumn{7}{c|}{ALP} \\
\cline{2-15}
& \multicolumn{4}{c}{Computational Time\cellcolor[gray]{0.9}} & \multicolumn{3}{|c||}{Relative Gap (\%)} & \multicolumn{4}{c|}{Computational Time\cellcolor[gray]{0.9}} & \multicolumn{3}{c|}{Relative Gap (\%)} \\
\hline
\multicolumn{1}{ |c|  }{Jobs} 	& TO & Min 	& Avg 	& Max 	& Min 	& Avg 	& Max 	& TO 	& Min 	& Avg 	& Max 	& Min 	& Avg 	& Max \\
\hline
\multicolumn{1}{ |c|  }{10}  	&0	 &0.031	&2.592	&23.71	&0.0	&0.0	&0.0 	&0 		&0.016	&0.238	&0.546	&0.0	&0.0	&0.0 \\
\hline
\multicolumn{1}{ |c|  }{20}  	&14  &25.02	&3876 	&5204 	&0.0 	&34.1 	&63.1 	&1 		&0.312 	&406.6 	&3647	&0.0 	&0.1 	&1.1  \\
\hline
\multicolumn{1}{ |c|  }{30}  	&15	 &3600	&3831 	&4816 	&31.7 	&58.5 	&81.5 	&12 	&570.9 	&3284 	&4076	&0.0 	&1.0 	&2.1   \\
\hline
\multicolumn{1}{ |c|  }{40}  	&15	 &3601	&3684	&3869	&42.0	&66.8 	&84.2 	&15		&3602	&3951	&4524	&1.0 	&1.7 	&3.0   \\
\hline
\end{tabularx}}
\end{table}\\
%%%%%%%%%%%%%%%%%%%%%%%%%%
\begin{table}[h!] \small
\renewcommand{\arraystretch}{1.4} %defines space between rows
\caption{Group 3 - 3 to 5 jobs per customer on ALP model (60 instances)}
\label{tab:t7}
\centering {
\begin{tabularx}{\textwidth}{X|X|X|X|X|X|X|X||X|X|X|X|X|X|X|}
\cline{2-15}
& \multicolumn{7}{c||}{BLP} & \multicolumn{7}{c|}{ALP} \\
\cline{2-15}
& \multicolumn{4}{c}{Computational Time\cellcolor[gray]{0.9}} & \multicolumn{3}{|c||}{Relative Gap (\%)} & \multicolumn{4}{c|}{Computational Time\cellcolor[gray]{0.9}} & \multicolumn{3}{c|}{Relative Gap (\%)} \\
\hline
\multicolumn{1}{ |c|  }{Jobs} 	& TO & Min 	& Avg 	& Max 	& Min 	& Avg 	& Max 	& TO 	& Min 	& Avg 	& Max 	& Min 	& Avg 	& Max\\
\hline
\multicolumn{1}{ |c|  }{10}  	&1	 &0.172 &352.8 	&3605	&0.0	&1.7 	&25.9 	& 0		&0.032 	&0.148 	&0.343	&0.0	&0.0	&0.0 \\
\hline
\multicolumn{1}{ |c|  }{20}  	&14  &659.5	&3835 	&5004 	&0.0 	&29.2 	&68.3	& 0		&0.218 	&9.348 	&73.15	&0.0	&0.0	&0.0  \\
\hline
\multicolumn{1}{ |c|  }{30}  	&15  &3604 	&4087	&5073 	&27.4 	&50.9 	&85.3	& 1		&3.729 	&394.7 	&3603 	&0.1 	&0.1 	&0.1  \\
\hline
\multicolumn{1}{ |c|  }{40}  	&15  &3600 	&3722	&3922 	&53.2	&65.4 	&84.1	& 14	&1927	&3563 	&3978	&0.0 	&0.9 	&1.6   \\
\hline
\end{tabularx}}
\end{table}\\
%%%%%%%%%%%%%%%%%%%%%%%%%%
\begin{table}[h!] \small
\renewcommand{\arraystretch}{1.4} %defines space between rows
\caption{Group 4 - 5 to 8 jobs per customer on ALP model (60 instances)}
\label{tab:t8}
\centering {
\begin{tabularx}{\textwidth}{X|X|X|X|X|X|X|X||X|X|X|X|X|X|X|}
\cline{2-15}
& \multicolumn{7}{c||}{BLP} & \multicolumn{7}{c|}{ALP} \\
\cline{2-15}
& \multicolumn{4}{c}{Computational Time\cellcolor[gray]{0.9}} & \multicolumn{3}{|c||}{Relative Gap (\%)} & \multicolumn{4}{c|}{Computational Time\cellcolor[gray]{0.9}} & \multicolumn{3}{c|}{Relative Gap (\%)} \\
\hline
\multicolumn{1}{ |c|  }{Jobs} 	& TO & Min 	& Avg 	& Max 	& Min 	& Avg 	& Max 	& TO 	& Min 	& Avg 	& Max 	& Min 	& Avg 	& Max\\
\hline
\multicolumn{1}{ |c|  }{10}  	&4	 &0.062	&967.4	&3607 	&0.0  	&6.1 	&43.3	&0		&0.031	&0.174 	&0.562	&0.0  	&0.0	&0.0 \\
\hline
\multicolumn{1}{ |c|  }{20}  	&15	 &3604	&4003 	&5175 	&3.6  	&36.8 	&69.5	&0 		&0.14 	&1.466 	&5.616 	&0.0	&0.0	&0.0  \\
\hline
\multicolumn{1}{ |c|  }{30}  	&15  &3603	&3901 	&4783 	&31.6	&64.5 	&81.1 	&0 		&1.014 	&98.52 	&1049 	&0.0	&0.0	&0.0   \\
\hline
\multicolumn{1}{ |c|  }{40}  	&15  &3609	&3730 	&3967 	&50.5	&67.8 	&79.0	&0		&0.175	&289.1	&2366	&0.0	&0.0	&0.0  \\
\hline
\end{tabularx}}
\end{table} 

In terms of the computational time, it can be seen from Table \ref{tab:t5} and  Table \ref{tab:t6} that for all the instances are solved optimally ($10$-Jobs and TO = 0). The ALP model is between 5 to 10 times faster than the BLP model. In terms of the relative gap -comparing the instances that timed-out for both models \footnote{This excludes the instances that are solved optimally by the ALP model and timed-out on the BLP model.}- it is observed that the relative gap reached by the ALP model in the worst of cases ($4.6\%$) is significantly smaller than the gap reached by the BLP model in the best of cases ($31.6\%$). See Figure \ref{fig:f2} for a visual summary of the relative gap data.
\begin{figure}[h!]
\caption{Percent relative gap of timed-out instances}
\label{fig:f2}
\centering
\includegraphics[angle=0,width=5in]{fig2.png}
\end{figure}







